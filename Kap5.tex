\label{Ch:Conclusiones}
Los reporte recientes de las anomalías en los decaimientos del mesón $B$ medidos en Babar, Belle y LHCb pueden ser explicados adicionando un bosón gauge $W'$ al ME, el cual preferencialmente se acople a la tercera generación de fermiones. En este trabajo se estudian las corrientes y posibles señales en la busqueda y detección futura del $W'$ en un modelo simplificado, en el cual el $W'$ solo se acopla a los quarks $b$ y $c$ y al sabor leptónico $\tau$, el cual es responsable de la corriente cargada $b \to c \tau \nu$ que explica los excesos en $R(D)$ y $R(D^*)$~\cite{Abdullah:2018ets}. Mediante los procesos de fusión de $g c$, $g g$ y $c b$, se puede producir un $W'$ que no se acople a la primera generaicón de  fermiones.

Estos nuevos procesos de producción del $W'$ pueden conducir a un jet $b$ junto con $\tau+\nu$ de estado final, lo cual nos permite usar un nuevo estado final exclusivo $b+\tau_h+E^{miss}_T$ reportado aquí para la busqueda del $W'$ en el LHC. El estudio detallado del background y la señal muestran que la presencia de el jet $b$ en el estado final exclusivo es bastante efectivo para reducir el background del ME. Además, la aparición de un jet $b$ en el estado final establecería el acoplamiento del $W'$ con los fermiones de tercera generación, lo cual es crucial para dar explicación de la anomalía.

Los análisis inclusivos de CMS y ATLAS muestran una restricción para la masa del $W^\prime$ por encima de 300 GeV y 500 GeV, respectivamente. Sin embargo, utilizando los cortes desarrollados en nuestro estudio demostramos que el alcance puede mejorarse para el rango de masa 200-500 GeV. Igualmente, se muestra que el rango de masas de 250-500 GeV puede probarse a nivel de $\sim 5 \sigma$ con una luminosidad de 100 fb$^{- 1}$ para $g_b = 0.1$ y $g_\tau = 0.1$. Consecuentemente, una región amplia del espacio de parámetros que explica las anomalías puede ser investigada directamente en el LHC. Los resultados del análisis aquí presentado pueden ser aplicados a un modelo con un $W'$ con canales de decaimiento adicionales después de aplicar un factor de escala adecuado, derivado de los branching de $W'$. 