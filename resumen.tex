\begin{abstract}

En éste trabajo se estudia la sensitividad en el LHC de la producción de un $W^{\prime}$ via quarks bottom ($b$) y quarks charms ($c$), y decaimientos al sabor leptónico $\tau$ en el rango de masas de $200$ a $1000$ GeV. Aquí se muestra que el quarks $b$ necesario en el mecanismo de producción puede  mejorar la diferenciación de las señales respecto al background, en comparación con un análisis inclusivo que depende exclusivamente del  $\tau$-tagging y $E^{\text{miss}}_T$. Se presentan límites posibles en los acoplamientos y se comparan con el mejor ajuste a las anomalías en $R (D^{(*)})$ en las desintegraciones del mesón $B$. \\

   \begin{keywords}
    	LHC, Fisica de altas energías, Boson $W^{\prime}$, Anomalías de sabor.
   \end{keywords}

\end{abstract}
