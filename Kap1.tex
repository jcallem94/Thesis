\chapter{Introducci\'{o}n}
El Modelo Estándar de partículas (ME) es una teoría altamente exitosa, debido a que explica una gran cantidad de experimentos de altas energías, tanto en la frontera de energías, como de intensidad. Sin embargo, es una teoría que se considera incompleta, debido a la aparición de nuevos fenómenos los cuales no son posibles de explicar mediante este (materia oscura, energía oscura, masa de neutrinos, entre otros). Por lo que se espera que las manifestaciones de la Nueva Física (NF) o física más allá del ME, aparezcan por encima de la escala de Teraelectronvoltios (TeV), ya que hasta el momento no se ha encontrado nada a escalas menores. 

Con el inicio de operaciones del gran colisionador de hadrones (LHC, por sus siglas en inglés) a su máxima potencia, se esperaba encontrar evidencia que pudiera demostrar la existencia de nuevas partículas e interacciones que dieran solución a estos problemas, pero hasta el momento no se ha encontrado pruebas fuertes que den indicios sobre NF.

Una gran cantidad de teorías que explican NF consideran que el grupo gauge del ME $SU(3)_C \times SU(2)_L \times U(1)_Y$ está incluido en un grupo gauge más grande que al romper espontáneamente las simetrías (directamente o a través de varias etapas) cae en el grupo gauge del ME. El rompimiento espontaneo de simetrías de este nuevo grupo gauge se da en o sobre la escala de TeV, por lo que el ME es visto como un modelo efectivo válido a bajas energías. Algunos modelos que incluyen NF son: Teorías de Gran Unificación (TGU) y modelos inspirados en teorías de cuerdas~\cite{Faber:2018qon}. Curiosamente, se predicen una gran cantidad de observables cuando la última ruptura espontánea del grupo gauge extendido se produce. 

En particular, es posible poner a prueba estos modelos mediante observables de física de sabor, debido a la impresionante precisión y alcance de los experimentos actuales. Es tal la precisión, que gracias a estos se ha logrado encontrar algunas anomalías que van en contra de algunas predicciones del ME~\cite{Amhis:2014hma}. Una de las predicciones del ME es la Universalidad Leptónica (UL), ésta establece que los bosones vectoriales se acoplan por igual a las tres familias de leptones~\cite{Abazov:2007wy}. Lo que implica que dichos bosones deben decaer con igual probabilidad a las tres familias de leptones. 

Las mediciones de los decaimientos $b \rightarrow c \ell \nu$ para diferentes leptones de estado final, pueden ser utilizadas para probar la UL con una gran precisión, dado la cancelación de muchas fuentes de incertidumbres teóricas que ocurren en relaciones, tales como:
\begin{align}
R(D^{(*)}) = \frac{\Gamma(B \rightarrow D^{(*)} \tau \nu)}{\Gamma(B \rightarrow D^{(*)} \ell \nu)}\,,
\end{align}
con $\ell = e$ o $\mu$. BaBar, Belle y LHCb han observado  anomalías en la UL. Los últimos promedios medidos para estos procesos son $R(D) = 0.397 \pm 0.049$ y $R(D^*) = 0.316 \pm 0.019$~\cite{Aaij:2014ora}, lo cual implica una desviación combinada de $4 \sigma$ respecto al valor del ME. Adicionalmente, las mediciones de la razón
\begin{align}
R_K = \frac{\Gamma(B \rightarrow K \mu ^+ \mu ^-)}{\Gamma(B \rightarrow K e^+ e^-)}\,,
\end{align}
realizadas por la colaboración LHCb muestran una desviación de $2,6 \sigma$ del valor registrado para el ME, $R_K = 0,745 ^{+0,090} _{-0,074} \pm 0,036$~\cite{Aaij:2014ora}. 

Si estos observables se convierten en una prueba de la no UL, sería lo que muchos físicos interpretan como la primera señal firme de la existencia de física más allá del ME en la escala electrodébil~\cite{Hiller:2003js,Guevara:2015pza,Bordone:2016gaq}.
Igualmente se han detectado anomalías en la distribución angular en las observaciones del decaimiento $b \rightarrow s \mu ^+ \mu ^-$. Ajustes globales realizados a $b \rightarrow s \ell ^+ \ell ^-$ por diferentes grupos muestran una buena concordancia y una explicación coherente para obtener NF de estos valores anómalos del ME con significancias alrededor de $4 \sigma$ ~\cite{Descotes-Genon:2013wba,Horgan:2013pva,Ghosh:2014awa,Hurth:2014vma,Altmannshofer:2014rta,Altmannshofer:2015sma,Descotes-Genon:2015uva,Hurth:2016fbr}. Mientras que en el caso de los observables de $b \rightarrow s \mu ^+ \mu ^-$ la cuestión de las incertidumbres hadrónicas todavía plantea algún debate~\cite{Beaujean:2013soa,Descotes-Genon:2015uva,Jager:2014rwa,Hurth:2014vma,Lyon:2014hpa,Ciuchini:2015qxb}. 

El esfuerzo de los físicos de partículas teóricos se ha centrado en el estudio de diferentes modelos que permitan explicar dichas anomalías en el decaimiento del mesón $B$. Principalmente se han desarrollado modelos que pueden explicar sólo una de las anomalías: ya sea $R(D^{(*)})$ o $R_K$. Los modelos que explican ambos conjuntos de anomalías son más escasos. Esto se debe a la dificultad de diferenciar las contribuciones de procesos que poseen un tamaño similar y  que tienen lugar en el ME en órdenes diferentes: nivel de loop para $R_K$ y nivel de árbol para $R (D^{(*)})$.  En ese sentido es necesario estudiar modelos que permitan una posible explicación a ambas anomalías.

Una posible solución a las anomalías en el decaimiento del mesón $B$, son los modelos con dos dobletes de Higgs (2HDM, por sus siglas en ingles) tipo III  ~\cite{Chen:2017eby,Iguro:2017ysu}. En los modelos tipo III ambos dobletes de Higgs se acoplan a todos los campos fermiónicos, lo que permite corrientes neutras que cambian el sabor a nivel árbol. Igualmente, este modelo permite obtener un Higgs cargado después de rompimiento espontáneo de simetría. Dicho estado es utilizado como una corriente cargada, mediante la cual se da explicación a las anomalías expuestas anteriormente. Además, es posible establecer que el estado de Higgs cargado decaiga en leptones cargados y neutrinos, permitiendo posibles señales de detección en aceleradores modernos~\cite{AristizabalSierra:2006ri}.
Otra posible solución consiste en explicar las anomalías en términos de bosones de gauge extras a los del ME~\cite{Boucenna:2016wpr}. En este caso se adicionan bosónes vectoriales $Z^{\prime}$ y $W^{\prime}$, considerando que el grupo gauge del ME está embebido en un grupo gauge mayor, de la forma $SU(3)_C \times SU(2)_1 \times SU(2)_2 \times U(1)_Y$. Gracias a que estos bosones gauge se pueden producir en los aceleradores del LHC, es posible establecer restricciones de los acoples de estos directamente a los quarks de la primera generación. Estos acoples suelen tener fuertes restricciones con los datos actuales. Sin embargo, si dichos bosones se acoplan poco a las partículas de las primeras generaciones, esto es favorable, ya que las anomalías de sabor usualmente están asociadas con partículas de la tercera generación.

En este trabajo nos centraremos en tratar de dar explicación a las anomalías de sabor mediante la introducción de un nuevo bosón vectorial cargado $W^{\prime}$ al ME. Este bosón permite dar explicación a las diferentes anomalías con una corriente cargada implicada, por lo que es necesario generar nuevas señales de producción de  dicho bosón que puedan ser observados en experimentos actuales de física de altas energías. Por lo que el objetivo principal será desarrollar nuevas técnicas de búsqueda, que permitan dilucidar dichas anomalías. 

El resto de este documento está organizado de la siguiente manera: En el capítulo \ref{Ch:W} se describen las propiedades más importantes del bosón $W$. En el capítulo \ref{Ch:W-prime} se describe las interacciones del bosón $W^{\prime}$ y sus propiedades más relevantes. En el capítulo \ref{Ch:results} se presentan los resultados y se hace un analisis de estos. En el capítulo \ref{Ch:Conclusiones} se presentan las conclusiones y resumen de los resultados.
